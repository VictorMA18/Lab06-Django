\documentclass[10pt, a4paper]{article}

\usepackage[utf8]{inputenc}
\usepackage[english,spanish]{babel}
\usepackage[left=25mm, right=25mm, top=35mm, bottom=30mm, headheight=35mm]{geometry}
\usepackage{graphicx}
\usepackage{xcolor}
\usepackage{fancyhdr}
\usepackage{hyperref}
\usepackage{setspace}
\usepackage{float}
\usepackage{listings}
\lstset{
  basicstyle=\ttfamily,
  keywordstyle=\color{blue},
  commentstyle=\color{green},
  stringstyle=\color{red},
  showstringspaces=false,
  breaklines=true,
  frame=single,
  backgroundcolor=\color{lightgray},
  columns=fullflexible
}
\lstdefinestyle{JavaStyle}{
    language=Java,
    basicstyle=\small\ttfamily,
    keywordstyle=\color{blue},
    commentstyle=\color{green},
    stringstyle=\color{purple},
    tabsize=4,
    showspaces=false,
    showstringspaces=false
} 
\lstdefinestyle{JavaScriptStyle}{
    language=JavaScript,
    basicstyle=\small\ttfamily,
    keywordstyle=\color{blue},
    commentstyle=\color{green},
    stringstyle=\color{purple},
    tabsize=4,
    showspaces=false,
    showstringspaces=false
} 

% Define background color
\definecolor{background}{HTML}{2E3440}

% Syntax customization
\usepackage{minted}
\usemintedstyle{nord}
\setminted{bgcolor=background}
\setminted{breaklines}

% Variables
\newcommand{\university}{Universidad Nacional de San Agustín de Arequipa}
\newcommand{\faculty}{Facultad de Ingeniería de Producción y Servicios}
\newcommand{\program}{Escuela Profesional de Ingeniería de Sistemas}
\newcommand{\semester}{2024 - A}
\newcommand{\course}{img/web_programming}
\newcommand{\topic}{img/Django.jpg} 
\newcommand{\professor}{Carlo Jose Luis Corrales Delgado}
\newcommand{\students}{Mamani Anahua, Victor Narciso \\ Cuno Cahuari, Armando Steven} 
\newcommand{\github}{https://github.com/VictorMA18/Lab06-Django}
\newcommand{\video}{}
\newcommand{\mydate}{28 de mayo, 2024}

% Just parts and chapters numbered
\setcounter{secnumdepth}{0}

% Head and foot customization
\pagestyle{fancy}
\lhead{\raisebox{-0.2\height}{\includegraphics[width=4cm]{img/logo_unsa.png}}}
\chead{\fontsize{8}{8}\selectfont \university \\ \faculty \\ \textbf{\program}}
\rhead{\raisebox{-0.2\height}{\includegraphics[width=3.5cm]{img/logo_episunsa.png}}}
% \lfoot{Estudiante \student}
\lfoot{Semestre \semester}
\cfoot{}
\rfoot{Pág. \thepage}

\begin{document}

\begin{titlepage}
	\centering
	\includegraphics[width=14cm]{\course} \par
  \vfill \vfill
	\includegraphics[width=15cm]{\topic}\par
  \vfill \vfill
  {\textbf{Profesor(a):} \par}
	\professor \vfill
  {\textbf{Estudiantes:} \par}
	\students \vfill
  {\textbf{Repositorio GitHub:} \par}
  \href{\github}{\github} \vfill
  {\textbf{Video:} \par}
  \href{\video}{\video} \vfill
	{\large \mydate \par}
\end{titlepage}

\section{Clases en Models.py}
Este código define tres modelos en Django: Alumno, Curso, y Nota. El modelo Alumno tiene un campo nombre de tipo CharField con un máximo de 255 caracteres. El modelo Curso contiene un campo curso que es un CharField de hasta 100 caracteres y debe ser único. El modelo Nota establece relaciones con los modelos Alumno y Curso mediante claves foráneas (ForeignKey), y tiene tres campos nota, nota2, y nota3, todos ellos de tipo DecimalField con un máximo de cuatro dígitos y dos decimales, inicializados en 0. Además, cada modelo tiene un método str que retorna una representación legible del objeto.
\begin{figure}[H]
  \centering
  \includegraphics[width=0.7\textwidth]{img/imagen1.jpeg}
  \caption{Codigo de Models}
\end{figure} 

\section{Funciones en Views.py}
El código proporciona seis-funciones de vista en Django para gestionar un sistema escolar. La función crear-alumno(request) maneja la creación de nuevos registros de alumnos, mientras que lista-alumnos(request) recupera y muestra todos los registros de alumnos. Por otro lado, crear-curso(request) gestiona la creación de nuevos registros de cursos, y lista-cursos(request) muestra todos los cursos existentes. Además, crear-nota(request) permite la creación de nuevas notas, y lista-notas(request) muestra todas las notas registradas. Cada función valida y procesa los formularios correspondientes, redirigiendo a las páginas de lista respectivas después de guardar los datos en la base de datos. En conjunto, estas funciones proporcionan una interfaz completa para gestionar los datos de alumnos, cursos y notas en el sistema de notas.
\begin{figure}[H]
  \centering
  \includegraphics[width=0.7\textwidth]{img/imagen2.jpeg}
  \caption{Codigo de Views}
\end{figure} 

\begin{figure}[H]
  \centering
  \includegraphics[width=0.7\textwidth]{img/imagen3.jpeg}
  \caption{Codigo de Views}
\end{figure} 

\section{Funciones en Forms.py}
El código proporciona tres formularios en Django para los modelos de datos relacionados con un sistema escolar. El formulario AlumnoForm está asociado al modelo Alumno, permitiendo la creación de nuevos registros de alumnos con un campo para el nombre del alumno. CursoForm está vinculado al modelo Curso, facilitando la creación de nuevos registros de cursos. Por último, NotaForm se asocia al modelo Nota, ofreciendo campos para ingresar información sobre la nota de un alumno en un curso específico, incluyendo múltiples notas si es necesario. Cada formulario está diseñado para interactuar directamente con su respectivo modelo, simplificando el proceso de captura y almacenamiento de datos relacionados con el sistema escolar.
\begin{figure}[H]
  \centering
  \includegraphics[width=0.7\textwidth]{img/imagen4.jpeg}
  \caption{Codigo de Forms}
\end{figure}  

\section{Funciones en Urls.py en la aplicación}
El bloque de código define un conjunto de rutas en Django que asignan solicitudes HTTP a funciones de vista específicas dentro de la aplicación. Cada ruta, configurada mediante el método path(), está asociada a una función de vista correspondiente en el módulo de vistas. Por ejemplo, /crearAlumno/ dirige las solicitudes a la función crear-alumno, encargada de la creación de nuevos registros de alumnos, mientras que /listaAlumno/ muestra la lista de todos los alumnos registrados utilizando la función lista-alumnos. Además, /crearNota/ dirige las solicitudes a la función crear-nota para crear nuevas notas, y /listaNota/ muestra todas las notas registradas utilizando la función lista-notas. Este enfoque permite a los usuarios interactuar con diversas funcionalidades del sistema de notas, como la gestión de alumnos, cursos y notas, a través de URLs específicas y funciones de vista asociadas.
\begin{figure}[H]
  \centering
  \includegraphics[width=0.7\textwidth]{img/imagen5.jpeg}
  \caption{Codigo de Urls.py en la aplicación}
\end{figure} 

\section{Funciones en Urls.py en el proyecto}
El bloque de código configura las rutas principales de la aplicación Django. La ruta /admin/ dirige a la interfaz de administración predeterminada de Django, mientras que la ruta /notas/ es manejada por la aplicación colegio.urls, que a su vez incluye un conjunto de rutas definidas en el archivo urls.py de la aplicación colegio. Esto permite una estructura de URL modular, donde las rutas relacionadas con las notas (definidas en colegio.urls) se pueden gestionar de manera independiente de las rutas de administración u otras partes de la aplicación.
\begin{figure}[H]
  \centering
  \includegraphics[width=0.7\textwidth]{img/imagen6.jpeg}
  \caption{Codigo de Urls.py en el proyecto}
\end{figure} 

\section{Los archivos HTML}

\section{crear\_alumno.html }
\begin{figure}[H]
  \centering
  \includegraphics[width=0.7\textwidth]{img/imagen7.jpeg}
  \caption{Codigo de crearAlumno.html}
\end{figure}  

\begin{figure}[H]
  \centering
  \includegraphics[width=0.7\textwidth]{img/CrearAlumno.png}
  \caption{Pagina}
\end{figure}

\section{lista\_alumnos.html }
\begin{figure}[H]
  \centering
  \includegraphics[width=0.7\textwidth]{img/imagen10.jpeg}
  \caption{Codigo de listaAlumnos.html}
\end{figure} 

\begin{figure}[H]
  \centering
  \includegraphics[width=0.7\textwidth]{img/ListaAlumnos.png}
  \caption{Pagina}
\end{figure}

\section{crear\_curso.html }
\begin{figure}[H]
  \centering
  \includegraphics[width=0.7\textwidth]{img/imagen8.jpeg}
  \caption{Codigo de crearCurso.html}
\end{figure} 

\begin{figure}[H]
  \centering
  \includegraphics[width=0.7\textwidth]{img/CrearCurso.png}
  \caption{Pagina}
\end{figure}

\section{lista\_cursos.html }
\begin{figure}[H]
  \centering
  \includegraphics[width=0.7\textwidth]{img/imagen11.jpeg}
  \caption{Codigo de listaCurso.html}
\end{figure}   

\begin{figure}[H]
  \centering
  \includegraphics[width=0.7\textwidth]{img/ListaCursos.png}
  \caption{Pagina}
\end{figure}

\section{crear\_nota.html }
\begin{figure}[H]
  \centering
  \includegraphics[width=0.7\textwidth]{img/imagen9.jpeg}
  \caption{Codigo de crearNota.html}
\end{figure}   

\begin{figure}[H]
  \centering
  \includegraphics[width=0.7\textwidth]{img/CrearNota.png}
  \caption{Pagina}
\end{figure}
 

\section{lista\_notas.html }
\begin{figure}[H]
  \centering
  \includegraphics[width=0.7\textwidth]{img/imagen12.jpeg}
  \caption{Codigo de listaNota.html}
\end{figure} 

\begin{figure}[H]
  \centering
  \includegraphics[width=0.7\textwidth]{img/ListaNotas.png}
  \caption{Pagina}
\end{figure}

\section{Ejecutar el servidor}
El comando python manage.py makemigrations colegio se utiliza en Django para crear nuevas migraciones basadas en los cambios realizados en los modelos del aplicativo colegio. Las migraciones son archivos que describen cómo modificar la estructura de la base de datos para mantenerla sincronizada con los modelos de Django. Este comando analiza los modelos en la aplicación colegio y genera los archivos de migración necesarios para aplicar esos cambios a la base de datos.
\begin{lstlisting}[language=bash]
  python manage.py makemigrations colegio
  python manage.py migrate
\end{lstlisting}

El comando python manage.py runserver se utiliza en Django para iniciar el servidor de desarrollo local. Una vez ejecutado, el servidor se activa y permite acceder a la aplicación web en desarrollo a través de un navegador en la dirección http://localhost:8000/ por defecto. Es una herramienta fundamental durante el proceso de desarrollo de aplicaciones web con Django, ya que proporciona un entorno de prueba para probar y depurar el código antes de desplegar la aplicación en un entorno de producción.
\begin{lstlisting}[language=bash]
  python manage.py runserver
\end{lstlisting}

\section{Acceder a la aplicacion en el navegador}
Crear un alumno en \href{http://127.0.0.1:8000/notas/crearAlumno/}{http://127.0.0.1:8000/notas/crearAlumno/}
\singlespacing
Crear un curso en \href{http://127.0.0.1:8000/notas/crearCurso/}{http://127.0.0.1:8000/notas/crearCurso/}
\singlespacing
Crear una nota en \href{http://127.0.0.1:8000/notas/crearNota/}{http://127.0.0.1:8000/notas/crearNota/}


\end{figure}
\item \textbf{URL de video de explicación:} \url{}
\item \textbf{URL de repositorio de GitHub:} \url{https://github.com/VictorMA18/Lab06-Django}
\end{document}


